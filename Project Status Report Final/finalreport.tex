
%\documentclass[10pt,proceedings]{IEEEtran}
%\documentclass[a4paper, 10pt, conference]{ieeeconf}      % Use this line for a4 paper

\documentclass[letterpaper, 10 pt, proceedings]{ieeetran}  % Comment this line out
%\IEEEoverridecommandlockouts                              % This command is only


\usepackage{geometry}
\usepackage{graphicx}
\graphicspath{{./Figures/}}
\usepackage{amsmath}
%\usepackage{amsfonts}
\usepackage{amsthm}
%\usepackage{stfloats}
\usepackage{caption}
\usepackage{float}
\usepackage{mathrsfs}
\usepackage{placeins}
\usepackage{subcaption}
\geometry{letterpaper,top=.75in,left=.69in,bottom=.75in,right=.69in}
\pagestyle{empty}


\newcommand{\A}{_{\textsc{\tiny \textit{A}}}}
\newcommand{\B}{_{\textsc{\tiny \textit{B}}}}
\newcommand{\D}{_{\textsc{\tiny \textit{D}}}}
\newcommand{\T}{_{\textsc{\tiny \textit{T}}}}
\newcommand{\G}{_{\textsc{\tiny \textit{G}}}}
\newcommand{\R}{_{\textsc{\tiny \textit{R}}}}
\newcommand{\sub}[1]{_\textsc{\tiny \textit{#1}}}
\newcommand{\sgn}{\text{sgn}}

\newtheorem{thm}{Theorem}


\title{\LARGE \bf
	Report 1: Examining the impact of antifragility and networked based decision making on the stock market
}

\author{Rachael Judy, Connor Klein, Josh Smith% <-this % stops a space
}


\begin{document}
	
	\maketitle
	\thispagestyle{empty}
	\pagestyle{empty}
	
%	\begin{abstract}
%		Using some of the theories from Taleb regarding antifragility, construct a network of firms and networked brokers inside firms to explore the impact of different preferences for risk as well as impact of weighting neighbors actions in decisions. 
%	\end{abstract}
	
	\section{Current Status}
	Research has been done on the topic of modeling the market as a complex network. Of the two common approaches for modeling the market, the team is taking the approach of modeling the market as a network of brokers. Papers that are being pulled from are noted in the references. Metrics such as Pearson Correlation Coefficient and mutual information rate for evaluating the relationships of stocks have been considered. Further, different methods of analyzing risk have been examined. The current thoughts are evaluating risk as a weighted combination of percentage of total investments, dividend versus value stocks, large cap and small cap ratios, and a risk reward ratio. Both the risk and the number of friends are modeled as fat tailed distributions to resemble the real world.
	
	
	Further, data has been collected and formatted to be selected by start date and rate of trades. The input for the data into the simulation is being examined.
	
	Additionally, a Broker class has been developed that includes functions for updating status every turn and evaluating the risk based on the influence factor from neighbors. This class has been added to networks and random influence weights assigned. This can be viewed in Figure \ref{samplegraph}. The edges of the graph are directed as influence of one broker on another could go either way and are weighted according to the influence this broker has. The influence factor has been decided to affect the risk tolerance for individual stocks. If the neighbors report a lower value of risk for a certain stock, the Broker lowers their assessment of the risk for that stock. Further, the number of connections per broker is determined by a power law distributions. The influence is initially set to be uniform and set to shift based on performance of agents.
	
	\begin{figure}
		\includegraphics[width=.5\textwidth]{samplegraph.png}
		\caption{Sample graph generated}
		\label{samplegraph}
	\end{figure}
	
	
	
%
	
	\bibliographystyle{ieeetran}
	\begin{thebibliography}{99}	
		\bibitem{antifragile}
		N. N. Taleb, Antifragile: Things that gain from disorder. Harlow, England: Penguin Books, 2013.
		
		\bibitem{networkscrowdsmmarkets}
		D. Easley and J. Kleinberg, Networks, crowds, and markets: Reasoning about a highly connected world. Cambridge, England: Cambridge University Press, 2012.
		%- https://www.cs.cornell.edu/home/kleinber/networks-book/, particularly chapters 9-12, 17, 22
		
		\bibitem{neuralnets}
		A. Thakkar and K. Chaudhari, “A comprehensive survey on deep neural networks for stock market: The need, challenges, and future directions,” Expert Systems with Applications, vol. 177, p. 114800, Sep. 2021, doi: 10.1016/j.eswa.2021.114800.
		%- https://www.sciencedirect.com/science/article/pii/S0957417421002414
		
		\bibitem{complexmacroeconomics}
		C. Hommes, “Behavioral and Experimental Macroeconomics and Policy Analysis: A Complex Systems Approach,” Journal of Economic Literature, vol. 59, no. 1, pp. 149–219, Mar. 2021, doi: 10.1257/jel.20191434.
		%- https://pubs.aeaweb.org/doi/pdfplus/10.1257/jel.20191434
		
		\bibitem{networkstocks}
		C. K. Tse, J. Liu, and F. C. M. Lau, “A network perspective of the stock market,” Journal of Empirical Finance, vol. 17, no. 4, pp. 659–667, Sep. 2010, doi: 10.1016/j.jempfin.2010.04.008.
		%- https://www.sciencedirect.com/science/article/pii/S0927539810000368
		
		\bibitem{marketcrasheswikipedia}
		“List of stock market crashes and bear markets,” Wikipedia. Jan. 24, 2024. Accessed: Feb. 13, 2024. [Online].
		%- dates noted as stock crashes (https://en.wikipedia.org/wiki/List_of_stock_market_crashes_and_bear_markets)
		
		\bibitem{antifragile}
		N. N. Taleb, Antifragile: Things that gain from disorder. Harlow, England: Penguin Books, 2013.
		
		\bibitem{networkscrowdsmmarkets}
		D. Easley and J. Kleinberg, Networks, crowds, and markets: Reasoning about a highly connected world. Cambridge, England: Cambridge University Press, 2012.
		%- https://www.cs.cornell.edu/home/kleinber/networks-book/, particularly chapters 9-12, 17, 22
		
		\bibitem{neuralnets}
		A. Thakkar and K. Chaudhari, “A comprehensive survey on deep neural networks for stock market: The need, challenges, and future directions,” Expert Systems with Applications, vol. 177, p. 114800, Sep. 2021, doi: 10.1016/j.eswa.2021.114800.
		%- https://www.sciencedirect.com/science/article/pii/S0957417421002414
		
		\bibitem{complexmacroeconomics}
		C. Hommes, “Behavioral and Experimental Macroeconomics and Policy Analysis: A Complex Systems Approach,” Journal of Economic Literature, vol. 59, no. 1, pp. 149–219, Mar. 2021, doi: 10.1257/jel.20191434.
		%- https://pubs.aeaweb.org/doi/pdfplus/10.1257/jel.20191434
		
		\bibitem{networkstocks}
		C. K. Tse, J. Liu, and F. C. M. Lau, “A network perspective of the stock market,” Journal of Empirical Finance, vol. 17, no. 4, pp. 659–667, Sep. 2010, doi: 10.1016/j.jempfin.2010.04.008.
		%- https://www.sciencedirect.com/science/article/pii/S0927539810000368
		
		\bibitem{marketcrasheswikipedia}
		“List of stock market crashes and bear markets,” Wikipedia. Jan. 24, 2024. Accessed: Feb. 13, 2024. [Online].
		
		\bibitem{kulmann}
		M. Kuhlmann, "Explaining financial markets in terms of complex systems," \textit{Philosphy of Science}, vol. 81, no. 5, pp. 1117-1130, Dec. 2014. %doi: 10.1086/677699.
		
		\bibitem{dimaggio_relevancebrokernetworks}
		M. Di Maggio, F. Franzoni, A. Kermani, and C. Sommavilla, "The relevance of broker networks for information diffusion in the stock market," \textit{Journal of Financial Economics}, vol. 134, no. 2, pp. 419-446, Nov. 2019. %doi: 10.1016/j.jfineco.2019.04.002. 
		
		
		\bibitem{gai_contagion}
		P. Gai and S. Kapadia, "Contagion in financial networks," \textit{Proceedings of the Royal Society}, vol. 466, no. 2120, pp. 2401–2423, Aug. 2010.
		
		\bibitem{liu_networkperspective}
		C. K. Tse, J. Liu, and F. C. M. Lau, “A network perspective of the stock market,” \textit{Journal of Empirical Finance}, vol. 17, no. 4, pp. 659–667, Sep. 2010. % doi: 10.1016/j.jempfin.2010.04.008.
		
		\bibitem{li_correlation}
		G. Li, A. Zhang, Q. Zhang, D. Wu, and C. Zhan, “Pearson correlation coefficient-based performance enhancement of broad learning system for stock price prediction,” \textit{IEEE Transactions on Circuits and Systems II: Express Briefs}, vol. 69, no. 5, pp. 2413–2417, May 2022. %, doi: 10.1109/TCSII.2022.3160266.
		
		\bibitem{fiedor_networksmutualinformationrate}
		P. Fiedor, “Networks in financial markets based on the mutual information rate,” \textit{Physical Review E}, vol. 89, no. 5, May 2014. % doi: 10.1103/PhysRevE.89.052801.
		
		\bibitem{cooper_realinvestmentandrisk}
		I. Cooper and R. Priestley. "Real investment and risk dynamics," \textit{Journal of Financial Economics}, vol. 101, no. 1, pp. 182-205, July 2011.
		
		\bibitem{lansing_riskaversion}
		K. J. Lansing and S. F. LeRoy, “Risk aversion, investor information and stock market volatility,” \textit{European Economic Review}, vol. 70, pp. 88-107, July 2014. 
		
		\bibitem{risktolerance}
		J. E. Corter and Y. J. Chen, “Do investment risk tolerance attitudes predict portfolio risk?,” \textit{Journal of Business and Psychology}, vol. 20, no. 3, pp. 369-381, 2006.
		
		\bibitem{harmon_economicinterdependence}
		D. Harmon, B. Stacey, Y. Bar-Yam, and Y. Bar-Yam, "Networks of economic market interdependence and systemic risk," New England Complex Systems Institute, Cambridge, MA, Tech. Report 1011.3707, Mar. 2009.
		
		\bibitem{park_complex}
		J. Park, C. H. Cho, and J. W. Lee, "A perspective on complex networks in the stock market," \textit{Frontiers in Physics}, vol. 10, Dec. 2022. %[Online serial] %, doi: 10.3389/fphy.2022.1097489.
		
		\bibitem{baydelli_hierarchicalmarket}
		Y. Y. Baydilli, S. Bayir, and I. Tuker, “A hierarchical view of a national stock market as a complex network,” \textit{Economic Computation \& Economic Cybernetics Studies \& Research}, vol. 51, no. 1, pp. 205–222, Jan. 2017.


	\end{thebibliography}
	
	
\end{document}
