\documentclass[12pt]{article}
\usepackage{amsmath}
\usepackage[margin=1in]{geometry}
\usepackage{enumitem}
\usepackage{float}
\usepackage{graphicx} % Required for inserting images
\usepackage{helvet}
\usepackage{hyperref}
\usepackage{listings}
\usepackage{mathptmx}
\usepackage{nameref}
\usepackage{pgfplots}
\usepackage{wrapfig}
\renewcommand{\familydefault}{\sfdefault}


\title{Complex Systems and Networks HW 1}
\author{Rachael Judy, Connor Klein, Josh Smith}

\begin{document}
\pgfplotsset{compat=1.18}
\setlist[itemize]{noitemsep}
\setlist[enumerate]{noitemsep}
 
\maketitle

\section{Q1: Annotated Bibliography}
% TODO: Write introduction tying together papers


\subsection{A network perspective of the stock market}
C. K. Tse, J. Liu, and F. C. M. Lau, “A network perspective of the stock market,” Journal of Empirical Finance, vol. 17, no. 4, pp. 659–667, Sep. 2010, doi: 10.1016/j.jempfin.2010.04.008.
\newline

% review
This paper presents network constructions of US stocks and proposes a new methodology for computing stock indices. The nodes of the networks consist of stocks, and the edges are established with a winner-take-all approach based on a threshold for cross correlation of daily stock prices, price returns, and trading volumes. The networks display a power-law distribution and suggest that the majority of stocks' time indices are strongly influenced by a small subset of stocks, particularly in the financial sector. They propose a new approach of using the networks for selection of stocks to represent the market indices. The paper highlights some flaws that may be present in the current methodology of determining stock indices and expands the importance of interconnectivity in examining the stock market. However, their methods are not examined over enough data to generalize the approach; the two windows examined by the authors are each two years in length and centered around the financial crisis of 2007-2008 which may not represent the overall market over time. Further, it does not examine the impact of adjusting several somewhat arbitrary parameters such as the threshold for creating an edge.


\subsection{Article 2}
% TODO: Citation and review

\subsection{Article 3}
% TODO: Citation and review

\subsection{Article 4}
% TODO: Citation and review

\subsection{Article 5}
% TODO: Citation and review

\subsection{Article 6}
% TODO: Citation and review

\subsection{Article 7}
% TODO: Citation and review

\subsection{Article 8}
% TODO: Citation and review

\subsection{Article 9}
% TODO: Citation and review


\section{Q2: ChatGPT Topic Description}
% TODO: insert sequence of prompts and responses


% TODO: discussion of results



\end{document}
